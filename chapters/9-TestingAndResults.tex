\chapter{Testing and Rsults}
\section{Introduction}
This chapter presents the testing and evaluation processes to validate the functionality, reliability and performance of the developed system.
The goal is to ensure that each component of the sytem works as intended. The system's components are tested individually and in integration to 
make sure thy meet the specified requirements.

For this project I tested the software using unit tests and integration tests. 
The unit tests are designed to test individual components of the system, while the integration tests are designed to test the system as a whole.
The hardware components are tested manually to ensure they work as expected. Results from these tests are documented to provide a clear 
overview of the system's performance and reliability.

\section{Testing}
\subsection{Unit Testing}
Unit testing is a software testing techniques where individual components are tested in isolation to ensure they function correctly.
I will present the unit tests that were created for the JwtService, which is a key component of the system, responsible of handling the JWT tokens.
Usually one unit test is created for each metho of the class. In this section I will present only the unit test for the GenerateJwtToken()
method. A unit test is split into three parts called Arrange, Act and Assert or Given, When and Then. In the Arrange part, the test
sets up the necessary conditions for the test, such as mocking and setting-up dependencies or creating test data. In the Act part, the test
calls the tested method with the prepared data. In the Assert part, the test checks if the result of the call is as expected.
\begin{lstlisting}[language=C, caption={Unit Test for GenerateJwtToken() Method in JwtService}, label={lst:jwt_unit_test}]
 [TestMethod ]
 public void GenerateJwtToken_ShouldReturnValidToken()
 {
     // Arrange
     var user = new User
     {
         Id = Guid.NewGuid(),
         Email = "test@example.com",
         FullName = "Test User",
         Role = UserRole.Admin
     };

     // Act
     var token = _jwtService.GenerateJwtToken(user);

     // Assert
     Xunit.Assert.False(string.IsNullOrWhiteSpace(token));

     // Parse token to validate structure
     var handler = new JwtSecurityTokenHandler();
     var readable = handler.CanReadToken(token);
     Xunit.Assert.True(readable);

     var jwtToken = handler.ReadJwtToken(token);
     Xunit.Assert.Equal("test@example.com", jwtToken.Payload["email"]);
     Xunit.Assert.Equal("Test User", jwtToken.Payload["name"]);
     Xunit.Assert.Equal("Admin", jwtToken.Payload[ClaimTypes.Role]);
 }
\end{lstlisting}

\subsection {Integration Testing}
Integration testing is a critical phase in the software development testing. In this phase, individual components are combined an tested 
together to ensure they work as a cohesive syste. This type of testing can be approached in various ways, such as top-down and bottom-up.
In this project, I used the top-down approach for integration testing. Manual integration testing was chosen for this project, due to the 
the presence of the hardware components, which are not easily testable with automated tests. Additionally, the system is relatively small and It
is at an early prototyping stage, where introducing automated tests would be an overhead.
The performed manual tests are presented in the following table.
\begin{table}[H]
\centering
\begin{tabular}{|p{1cm}|p{4cm}|p{6cm}|p{3cm}|p{2cm}|}
\hline
\textbf{ID} & \textbf{Test Name} & \textbf{Description} & \textbf{Expected Outcome} & \textbf{Result} \\
\hline
T1 & Sensor Data Flow & Trigger a real sensor reading on the ESP32 node. Ensure it is received by the gateway and forwarded to the backend API. & Data appears in the backend database with correct timestamp and values. & Pass \\
\hline
T2 & Frontend Data Display & After a new reading is received, open the Angular app dashboard. & UI shows latest sensor data. & Pass \\
\hline
T3 & Automatic Irrigation Trigger & Set a low moisture threshold and insert dry soil. System should automatically trigger irrigation. & Relay activates. & Pass \\
\hline
T4 & Manual Irrigation Control & From the frontend, press "Stop irrigation" button. & The relay should close. & Pass \\
\hline
T5 & System Recovery & Restart the ESP32 node and Raspberry Pi. & System resumes operation. & Fail \\
\hline
\end{tabular}
\caption{Manual Integration Test Plan}
\label{tab:manual_integration_test_plan}
\end{table}

As you can see, the system, in general, works as expected, excepting test T5, which failed. The system did not recover after a restart.
This is because the running process on the Raspberry Pi does not automatically start after a reboot. To fix this,
I need to add the main process to the crontab of the Raspberry Pi, so it starts automatically after a reboot.

\subsection{Performance Testing}
Performance testing is a type of testing that evaluates the system's performance under various conditions.
\subsubsection{End-to-End Data Transmission Time Test}
End-to-end data transmission time testing measures the time it takes for a sensor reading to be sent from the ESP32 to the server.
For this test, I added a new parameter to the sensor readings, which is the reading time, collected from the ESP32. When the reading is acknowledged
by the server, the server calculates the time it took for the reading to be sent and stored. Then for each reading the latency is calculated as
the difference between the reading time and the time it was received by the server.
The average latency is calculated over a number of readings, which is 100 in this case. 
I repeated this test in 3 different scenarios. First scenario is when the Raspbery Pi is connected to the local network via Ethernet cable,
the second scenario is when the Raspberry Pi is connected to the local network via Wi-Fi. The third scenario is when the Raspbbery Pi is connected to the internet
the Ethernet cable, and the and the ESP32 node is placed in a different location than the ESP32 gateway. 
Below are presented the test scenarios:
\begin{table}[H]
\centering
\begin{tabular}{|p{1.2cm}|p{5cm}|p{5cm}|p{5cm}|}
\hline
\textbf{S.} & \textbf{Description} & \textbf{Network Setup} & \textbf{Expected Impact on Latency} \\
\hline
1 & Raspberry Pi connected via Ethernet & ESP32 Gateway is connected to the Raspberry Pi via UART; and the Pi is wired to the router & Low and stable latency due to reliable Ethernet connection \\
\hline
2 & Raspberry Pi connected via Wi-Fi & ESP32 Gateway is connected to the Raspberry Pi via UART; Pi connects to the router over Wi-Fi & Slightly increased latency due to potential Wi-Fi interference or variability in wireless throughput \\
\hline
3 & LoRa Node communicates with remote Gateway, which is UART-connected to Raspberry Pi & ESP32 Node is in a different location and sends data over LoRa to the Gateway; the Gateway forwards data via UART to the Raspberry Pi & Higher latency caused by LoRa transmission time and distance; UART adds negligible delay, backend response depends on network setup \\
\hline
\end{tabular}
\caption{Sensor-to-Backend Latency Test Scenarios with UART-connected ESP32 Gateway}
\label{tab:latency_scenarios_uart}
\end{table}

The assumption is that the latency will be lower when the Raspberry Pi is connected to the local network via Ethernet cable.
The first 2 scenarios are expected to have similar results, and the third scenario is expected to have a higher latency due to the LoRa transmission
time and distance between the ESP32 Node and the Gateway. This would mean that the bottleneck is the LoRa transmission time, and not the UART transmission time,
Ethernet or Wi-Fi connection.

The results of the end-to-end data transmission time test are presented in Table \ref{tab:latency_results}.
\begin{table}[H]
\centering
\begin{tabular}{|c|c|c|c|}
\hline
\textbf{Scenario} & \textbf{Min Latency (ms)} & \textbf{Max Latency (ms)} & \textbf{Avg Latency (ms)} \\
\hline
1 & 110 & 145 & 127 \\
\hline
2 & 118 & 152 & 134 \\
\hline
3 & 340 & 460 & 392 \\
\hline
\end{tabular}
\caption{Measured Sensor-to-Backend Latency Across Scenarios}
\label{tab:latency_results}
\end{table}

The first 2 scenarios show similar result, with an average latency of around 130 ms, confirming that the Ethernet and Wi-Fi connections have similar
performance for this application. The third scenario shows a significantly higher latency of around 390 ms. The results confirm the assumption
that the bottleneck is the LoRa transmission time. The obtained performance is acceptable for the application, as the latency is within the range of 
a few hundred milliseconds. 

\subsection{Hardware Testing}
\subsubsection{Sensor Testing}
The sensors used in the system were tested to ensure they provide accurate readings.
The tests involved measuring the sensor readings in different conditions and comparing them to known values. The soil moisture sensor's VCC 
and GND pins were connected to the pins of a stable power source of 3.3V and the voltmeter's positive and negative probes were connected
to the sensor's output pin and GND pin respectively. Then the sensor was placed in different soil conditions (dry, wed)
and the voltage readings were recorded. The result are presented in Table
The rain sensor was tested in a similar manner, by placing it in different conditions.
The DHT111 sensor could not be tested with a voltmeter, as it provides digital readings. For this sensor I used 
the ESP32 to read the sensor data and display it on the serial monitor. I compared the reading with the values from a thermometer and hydrometer. The
result were not quite perfect, but they were within the acceptable range.
The results of the manual sensor testing are presented in the following table:
\begin{table}[H]
\centering
\begin{tabular}{|p{4.2cm}|p{5cm}|p{3.5cm}|p{3.5cm}|}
\hline
\textbf{Sensor} & \textbf{Condition Tested} & \textbf{Measured Voltage (V)} & \textbf{Interpretation} \\
\hline
Soil Moisture Sensor & Inserted in wet soil & 2.8 – 3.3 V & High voltage → soil is wet (low resistance) \\
\hline
Soil Moisture Sensor & Inserted in dry soil & 0.8 – 1.2 V & Low voltage → soil is dry (high resistance) \\
\hline
Rain Sensor & Plate completely dry & 0.5 – 1.0 V & Low voltage → no rain detected \\
\hline
Rain Sensor & Plate with water droplets & 2.9 – 3.3 V & High voltage → rain detected \\
\hline
\end{tabular}
\caption{Manual Sensor Testing Using a Voltmeter}
\label{tab:voltmeter_sensor_test}
\end{table}