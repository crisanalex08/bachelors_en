\chapter{Introduction}\label{section:introduction}
\thispagestyle{pagestyle}


\section{CONTEXT}

Agriculture is a vital sector that plays a crucial role in sustaining human life and the economy.
Agriculture automation and optimization has become a major concern in recent years. 
As the global population continues to grow, the demand for food is increasing and the developing need for food
along with the effect of climate changes are forcing the agricultural industry to adapt and innovate\cite{OBAIDEEN2022100124}.

In the last 35 years, the wold has seen a doubling of the agricultural production. This has been achieved throught the use
of different fertilizers, pesticides and herbicides. This doubling was associated with a 6.87-fold increase in nitrogen fertilization,
a 3.48-fold increase in phosphorus fertilization and 1.68-fold increase in the amount of irrigated cropland \cite{doi:10.1073/pnas.96.11.5995}.
In addition, the water consumoption is expected to increase by 50\% by 2050 \cite{s20236865}. 

This project aims to address the challenges of water scarcity and the need of efficient irrigation systems
by presenting the plan, the implementation, the results and future work of a system that can be used in different
scenarios and is meant to help reducing the water consumption and increasing the agricultural productivity.
The PiIrrigate project intends achuive this by developing an innovative irrigation system that leverages the power of 
IoT and LoRa radio communication technologies.  
The main focus of this project is to create a system that can be easly used in different agricultural settings, 
starting from small gardens to large farms and even smart greenhouses. Beside this, I wanted to create a system that 
is easy to use and can be extended with ease. 

The ESP32 boards with sensors are responsible for colecting the data. Then data is sent using  LoRa to another
ESP32 board that acts as a gateway connected to a Raspberry Pi, which is responsible for sending the data to a web API.
The web API is developed in .NET and is responsible for storing the data in a PostgreSQL database.
The data is then sent to a web application using SignalR, which allows real-time communication between the server and the client.
The web application is responsible for displaying the live data and the historical data and also for providing a way to
control the system manually and to add new nodes to the system. 

This system takes advantage of the LoRa radio communication technology, 
which allows for long-range communication with low power consumption. Meaning that the system can be used in remote areas and
it will work even if the internet connection is not available to all the nodes. The Raspberry Pi is the only component of this system
that needs to be connected to the internet. Other components can be scatered on a area of 10km or more, depending on
the environment and the node setup (mesh or star topology).

\newpage
\section{MOTIVATION}


The reason why I choose to create such a system was fulled by my passion for technology and smart agriculture. Besides this,
I like to observe the data path, from the moment it is collected by the sensors, to the moment it is displayed
in a web application. 
I have always been intested in pieces of technology that can be used to solve real world problems and now I had the chance
to create such a system.

Initially, I wanted to create a system for my own lawn, but as I started working on the project, I realized that the system
can be used in many other scenarios, such as smart greenhouses or even large farms or vineyards. 

\chapter{State of the Art}\label{section:stateoftheart}

\section{Introduction}
The state of the art chapter provides an overview of the current state of smart irrigation 
systems and their applications in agriculture. 
This chapter will explore the existing technologies, methods, and solutions used in smart irrigation. 
It will also highlight the gaps and challenges in the current systems,
and how the PiIrrigate project aims to address these issues.

\section{Existing Smart Irrigation Solutions}
\subsection{Types of Smart Irrigation Systems}

There are several types of smart irrigation systems used in modern agriculture:

\begin{itemize}
  \item \textbf{Weather-Based Controllers} \\
  These systems use weather data to adjust irrigation schedules based on evapotranspiration rates, 
  ensuring that plants receive the right amount of water.

  \item \textbf{Soil Moisture-Based Controllers} \\
  These systems rely on data from soil moisture sensors placed in the root zone of the plants.
  Irrigation cycles are triggered when the soil moisture drops below a predetermined threshold, 
  ensuring plants receive water only when necessary.
  This method is very precise for specific zones \cite{smartIrrigationTechnologyControllersAndSensors}.

  \item \textbf{Hybrid Systems} \\
  Many modern systems utilize a hybrid approach, combining data from both
  weather feeds and soil moisture sensors for more accurate and resilient 
  irrigation decisions. Some research also explores "hybrid" in terms of 
  integrating different energy sources (e.g., solar and wind) to power the systems or 
  combining various 
  irrigation methods (like drip and sprinkler) under one smart control \cite{soilBasedIrrigation}.

  \end{itemize}