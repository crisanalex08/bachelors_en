\chapter{Introduction}\label{section:introduction}
\thispagestyle{pagestyle}


\section{CONTEXT}

Agriculture is a vital sector that plays a crucial role in sustaining human life and the economy.
Agriculture automation and optimization has become a major concern in recent years. 
As the global population continues to grow, the demand for food is increasing and the developing need for food
along with the effect of climate changes are forcing the agricultural industry to adapt and innovate\cite{OBAIDEEN2022100124}.

In the last 35 years, the wold has seen a doubling of the agricultural production. This has been achieved throught the use
of different fertilizers, pesticides and herbicides. This doubling was associated with a 6.87-fold increase in nitrogen fertilization,
a 3.48-fold increase in phosphorus fertilization and 1.68-fold increase in the amount of irrigated cropland \cite{doi:10.1073/pnas.96.11.5995}.
In addition, the water consumoption is expected to increase by 50\% by 2050 \cite{s20236865}. 

This project aims to address the challenges of water scarcity and the need of efficient irrigation systems
by presenting the plan, the implementation, the results and future work of a system that can be used in different
scenarios and is meant to help reducing the water consumption and increasing the agricultural productivity.
The PiIrrigate project intends achuive this by developing an innovative irrigation system that leverages the power of 
IoT and LoRa radio communication technologies.  
The main focus of this project is to create a system that can be easly used in different agricultural settings, 
starting from small gardens to large farms and even smart greenhouses. Beside this, I wanted to create a system that 
is easy to use and can be extended with ease. 

The ESP32 boards with sensors are responsible for colecting the data. Then data is sent using  LoRa to another
ESP32 board that acts as a gateway connected to a Raspberry Pi, which is responsible for sending the data to a web API.
The web API is developed in .NET and is responsible for storing the data in a PostgreSQL database.
The data is then sent to a web application using SignalR, which allows real-time communication between the server and the client.
The web application is responsible for displaying the live data and the historical data and also for providing a way to
control the system manually and to add new nodes to the system. 

This system takes advantage of the LoRa radio communication technology, 
which allows for long-range communication with low power consumption. Meaning that the system can be used in remote areas and
it will work even if the internet connection is not available to all the nodes. The Raspberry Pi is the only component of this system
that needs to be connected to the internet. Other components can be scatered on a area of 10km or more, depending on
the environment and the node setup (mesh or star topology).

\newpage
\section{MOTIVATION}


The reason why I choose to create such a system was fulled by my passion for technology and smart agriculture. Besides this,
I like to observe the data path, from the moment it is collected by the sensors, to the moment it is displayed
in a web application. 
I have always been intested in pieces of technology that can be used to solve real world problems and now I had the chance
to create such a system.

Initially, I wanted to create a system for my own lawn, but as I started working on the project, I realized that the system
can be used in many other scenarios, such as smart greenhouses or even large farms or vineyards. 
\subsection{Structure} \label{section:structure}
Each file belonging to the document can be found in \texttt{bachelors\_en/}.
The main file of the template is \texttt{main.tex}, which must be compiled using pdflatex (usually the default option). Files in \texttt{components/} define the template and \textbf{should not be modified}.

In \texttt{customs.tex} reside the personal data which appear in places not directly accessible by the user and should be modified. Every file in \texttt{chapters/} represents a chapter where the content of the thesis is written. Under \texttt{images/} will be placed all the images used in the document.

\subsection{Authenticity Declaration}
The blank declaration can be obtained in two ways:
\begin{itemize}
	\item from \texttt{main.tex} residing in \texttt{authenticity\_declaration/}, an unsigned variant is created after filling personal data in \texttt{customs.tex}, similarly with the structure of \cref{section:structure}.
	\item from the document at \url{https://ac.upt.ro/finalizare-studii/}, in section \enquote{Documente importante}. This should be manually filled with personal data.
\end{itemize}
Regardless of the option chosen, the document must be printed, \textbf{physically signed(with a pen)} and scanned in order to be attached. In \texttt{customs.tex}, the path must be changed from the example to the signed and scanned PDF document, which will be saved in \texttt{bachelors\_ro/}.

\subsection{Other \LaTeX\ elements}
References to other elements in the document (excepting citations and links): \cref{section:introduction}; \cref{fig:myfig}.

For each reference, a \texttt{\textbackslash label} is needed, which will be inserted anywhere in the text or immediatley after chapter/section titles, and in figures/tables/code fragments.
\\[\baselineskip]

Citations will be added using bibtex, example: \cref{example:citation}.
Inserting images (\cref{fig:myfig}), tables (\cref{table:table1}) and code fragments (\cref{code:polym}) will be added using the corresponding code. For tables, an external formatting tool is recommended: (example \cref{example:table_url}).

\section{General Information}

Each chapter must have a clear structure, will begin on a new page and will have a title. It will be followed by two blank 12 pt. lines.

Each sub-section title (e.g. 1.2 General information) shall begin after a 12 points blank line after the text and shall be followed by a 12 points blank line.

The text of the paper will be justified. It is recommended to check the spelling of the text with the help of the speller of the Word program. It is recommended that the thesis paper should not exceed a number of 100 pages, annexes included.


Rules applying to the text of the paper:
\begin{enumerate}[leftmargin=2cm,topsep=1.15pt,itemsep=1.15pt,partopsep=1.15pt,parsep=1.15pt,label=\alph*.]
   \item Margins of the page – the following values will be used:
   \begin{itemize}[topsep=1.15pt,itemsep=1.15pt,partopsep=1.15pt,parsep=1.15pt]
     \item interior: 2 cm 
     \item exterior: 2 cm 
     \item up: 2,5 cm (header included)
     \item down: 2 cm
   \end{itemize}
   \item Line spacing – the text shall apply a 1.15 line spacing 
   \item Indentation – text within regular paragraphs will be aligned between the left and the right margins (justified). The first line of each paragraph will have a 1.5 cm indentation. There will be an exception for chapter titles, which will be aligned left, just like the titles of the tables and of the figures (see explanations below);
   \item Font – the 12 points Nimbus Sans font will be used, as well as the specific diacritics of the paper language (ex: ă, ş, ţ, î, â - for the Romanian language);
   \item Page numbering -  Page numbering runs from the title page to the last page of the paper, but the page number appears starts on the Introduction page. The page number is inserted at the bottom of the page, centered.
   \item f.	Page header – it shall be inserted starting with the introduction page and contains, in successive lines, the following text, aligned left and with the size of 8 points: (i) the text Politehnica University of Timișoara ; (ii) the name of the program of study and the year of the paper defense; (iii) the name of the candidate (left) and the title of the paper. At the right of the header, the UPT logo may be inserted; 
\end{enumerate}