\chapter{Introduction}\label{section:introduction}
\thispagestyle{pagestyle}


\section{Context}

Agriculture is a vital sector that plays a crucial role in sustaining human life and the economy.
Agriculture automation and optimization has become a major concern in recent years. 
As the global population continues to grow, the demand for food is increasing and the developing need for food
along with the effect of climate changes are forcing the agricultural industry to adapt and innovate\cite{OBAIDEEN2022100124}.

In the last 35 years, the wold has seen a doubling of the agricultural production. This has been achieved throught the use
of different fertilizers, pesticides and herbicides. This doubling was associated with a 6.87-fold increase in nitrogen fertilization,
a 3.48-fold increase in phosphorus fertilization and 1.68-fold increase in the amount of irrigated cropland \cite{doi:10.1073/pnas.96.11.5995}.
In addition, the water consumoption is expected to increase by 50\% by 2050 \cite{s20236865}. 

This project aims to address the challenges of water scarcity and the need of efficient irrigation systems
by presenting the plan, the implementation, the results and future work of a system that can be used in different
scenarios and is meant to help reducing the water consumption and increasing the agricultural productivity.
The PiIrrigate project intends achuive this by developing an innovative irrigation system that leverages the power of 
IoT and LoRa radio communication technologies.  
The main focus of this project is to create a system that can be easly used in different agricultural settings, 
starting from small gardens to large farms and even smart greenhouses. Beside this, I wanted to create a system that 
is easy to use and can be extended with ease. 

The ESP32 boards with sensors are responsible for colecting the data. Then data is sent using  LoRa to another
ESP32 board that acts as a gateway connected to a Raspberry Pi, which is responsible for sending the data to a web API.
The web API is developed in .NET and is responsible for storing the data in a PostgreSQL database.
The data is then sent to a web application using SignalR, which allows real-time communication between the server and the client.
The web application is responsible for displaying the live data and the historical data and also for providing a way to
control the system manually and to add new nodes to the system. 

This system takes advantage of the LoRa radio communication technology, 
which allows for long-range communication with low power consumption. Meaning that the system can be used in remote areas and
it will work even if the internet connection is not available to all the nodes. The Raspberry Pi is the only component of this system
that needs to be connected to the internet. Other components can be scatered on a area of 10km or more, depending on
the environment and the node setup (mesh or star topology).

\newpage
\section{Motivation}

The reason why I choose to create such a system was fueled by my passion for technology and smart agriculture. Besides this,
I like to observe the data path, from the moment it is collected by the sensors, to the moment it is displayed
in a web application. 
I have always been intested in pieces of technology that can be used to solve real world problems and now I had the chance
to create such a system.
Initaially, I wanted to create a smaller system for my own graden, but then I reazied that this system can be used in a lerger
scale and can be adapted to different scenarios.

This project is also motivated by the opportunity to apply core concepts of computer science, such as API design, data management, 
distributed systems and IoT \- to build a cost-effective ans scalable smart irrigaiton system. This project intends to provide
not only a system for irrigation, but also a platform that can be used in different scenarios, such as smart greenhouses,
meteorological stations, or even smart cities.

\section{Objectives}
The main objective of this project is to develop a system that leverages IoT, cloud services, and modern softwware development
principles to optimize irrigation processes and reduce the water usage in agriculture. The focus of this project is to provide
a theoretical approach of the system, but also to present the implementation details and the results of the system. At the and 
of this project the result should comprise the system architecture, including the hardware and software components, and 
the implementation details of the system. The implementation details will include the hardware and software components used,
the communication protocols used, the data management and storage, and the user-interface design.
To be more specific, the objectives of this project are:
\begin{itemize}
    \item Develop a cost-effective and scalable smart irrigation system that can be used in different agricultural settings.
    \item Leverage IoT and LoRa radio communication technologies to create a system that can operate in remote areas with low power consumption.
    \item Provide a user-friendly web application for real-time monitoring and control of the irrigation system.
    \item Implement a modular architecture that allows for easy extension and customization of the system.
    \item Ensure data security and privacy by implementing best practices in API design and data management.
\end{itemize}

\section{Scope and Limitations}

Scope of this project includes the design, implementation, and evaluation of a smart irrigaiton system. 
The system will be designed to collect data from enviromental sensors, such as soil moisture, temperature, and humidity,
process this data throught a web API, and provide a user friendly web application for real-time monitoring and control.

Key fetures of the system include:
\begin{itemize}
    \item Data collection from environmental sensors using LILYGO Meshtastic AXP2101 T-Beam V1.2 ESP32 LoRa boards.
    \item Long-range communication using LoRa radio technology.
    \item Data processing and storage using a web API and PostgreSQL database.
    \item Real-time monitoring and control through a web application.
    \item Modular architecture for easy extension and customization.
\end{itemize}
Despite the comprehensive scope of this project, there are some limitations that need to be considered:
\begin{itemize}
    \item Hardware limitations: The system is designed to work with specific components, any changes to
     the hardware may require significant changes in the project.
    \item This specific system will be tested in a controlled environment, so the result may not be representative of 
    real-world scenarios. 
    \item Reliability and calibration of the hardware components are assumed to be optimal, but they are not guaranteed.
    \item Weather prediction and other external factors are outside the scope of this project,
    the system will not take into account weather forecasts.
    \item Energy consumption and power management analysis are not included in the scope of this project.
    \item The UI design is focused on functionality and usability, but it may not be visually appealing or optimized for all devices.
    \item Machine learning and advanced data analytics are not included in the scope of this project.
\end{itemize}

\section{Methodology overview}

The metodology adopted for this project is structured around the typical sofware development lifecycle,
beggining with the requirements gathering and analysis, followed by the design, implementation, testing, and evaluation phases.
This approach ensures that the system is developed in a systematic and organized manner, allowing for the creation 
of a robus and reliable smart irrigation system.

System requirements were gathederd based on on a research of modern irrigaiton practicies, with a great focus on the water consumoption
usage and the need of real-time monitoring and control. These requirements helped identifying core functionalities of the system,
such as data collection from environmental sensors, processing, storage, and user interface design. 

The system architecture was designed in a layered manner. The hardware layer includes the sensors, the ESP32 boards
and the Raspberry Pi. It is responsible for data collection. The communication layer includes all protocols and technologies
used for data transmission, such as LoRa, MQTT and HTTP. The data processing layer includes the IoT Hub, the webApi and the PostgreSQL database.
Finally the user interface layer includes the web application that provides real-time monitoring and control of the system. 