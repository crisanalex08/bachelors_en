\chapter{Implementation}
\section{Overview}
This chapter describes the implementation of the PiIrriate system. If the previous chapters described the architecture and 
the main components of the system, this chapter focuses on how these components are implemented in practice.
The implementation is divided into several sections, each focusing on a specific aspect of the system.

\section {ESP32 Firmware}
For the ESP32 development I had to choose between two main frameworks: ESP-IDF and Arduino. I
chosed the Arduino framework because it is more user-friendly. As IDE I used Platform IO, which is a VSCode extention
that allow to develop for many different platforms, including the ESP32\cite{platformio_docs}.

\subsection{Data Acquisition}
The data acquisition is done using the ESP32 nodes that collect data from the sensors.
For each sensor type, I created a specific library that can be used to read the data from the sensor.
Temperature and humidity data is collected using a DHT11 sensor. The communication with the sensor is done using
1-wire digital interface which involves 3 main steps\cite{1wire}:
\begin{enumerate}
  \item The microcontroller initiates communication by sending the start signal.
  The start signal is an 18\,ms LOW signal followed by a $20$--$40\,\mu$s HIGH signal.
  \item The sensor responds a fixed LOW and HIGH handshake pattern, indicating that it is ready to send data.
  Usually the acknowledgment is a $80\,\mu$s LOW signal followed by a $80\,\mu$s HIGH signal.
  \item After the handshake, the sensor sends a 40-bit data stream, which includes the humidity and temperature data.
  The bits are sent in a specific order: first the humidity data (16 bits), 
  then the temperature data (16 bits), 
  and finally a checksum (8 bits).
  Each bit is sent as a $50\,\mu$s LOW signal followed by a HIGH signal that lasts for either $26$--$28\,\mu$s (for a 0 bit) or $70\,\mu$s (for a 1 bit).
  In code, for each bit, the microcontroller waits for the LOW signal to start, 
  then waits for $30\,\mu$s then ig the signal is HIGH, the bit is a 1, otherwise it is a 0.
  The checksum is used to verify the integrity of the data received from the sensor.
\end{enumerate}

\begin{figure}[H]
    \centering
    \includegraphics[width=0.7\textwidth]{images/dht-steps.png}
    \caption{Steps in data aquisition from DHT11 sensor}
    \label{fig:dht-steps}
\end{figure}

For the soil moisure data acquisition, I used a resistive soil moisture sensor. The principle of operation is based
on measuring the resistance of the soil. The sensor consists of two probes that are inserted into the soil.
When the soil is dry, the resistance between the probes is high, and when the soil is wet, the resistance is low\cite{s20020363}.
Then an ADC is used to measure the voltage across the probes, which is transofmerd into digital value. In this case, the
ADC is a 12-bit ADC, which means that the digital value can range from 0 to 4095.

\begin{figure}[H]
    \centering
    \includegraphics[width=0.5\textwidth]{images/moisture-sensor.png}
    \caption{Soil moisture sensor based on resistive principle}
    \label{fig:moisture-sensor}
\end{figure}

The raindrop data is collected using a resistive raindrop sensor. The principle of operation is similar to the soil moisture sensor, but 
in this case, the sensor consists of a plate with two conductive traces that are separated by a small gap.
When the rain falls on the plate, it closes the gap and allows the current to flow between the traces. The 
is connected to an ADC which measures the voltage across the traces.

\begin{figure}[H]
    \centering
    \includegraphics[width=0.4\textwidth]{images/rain-detector-sensor.png}
    \caption{Raindrop sensor based on resistive principle}
    \label{fig:raindrop-sensor}
\end{figure}

Since the water consumption is a very important aspect in agriculture, 
I wanted to add a water flow sensor to the system. For the purpose of this project I used an 
YF-S201 water flow sensor, which is a low-cost sensor that can be used to measure the flow rate of water in a pipe.
The sensor consists of a plastic body with a 
turbine inside that rotates when water flows through it.
The rotation of the turbine generates a pulse signal that can be used to measure the 
flow rate of water.
The sensor has a maximum flow rate of 30 liters per minute and a minimum 
flow rate of 1 liter per minute.
The sensor has three wires: red (VCC), black (GND), and yellow (signal).
At each rotation of the tubine, the sesnsor generates a pulse signal on the signal wire.
This sensor will be used to check if the irrigation system is working properly or not.
The ESP32 counts the number of pulses in a given time interval 
(e.g., 1 second) to calculate the flow rate.
The flow rate can be calculated using the following formula:
\begin{equation}
    \text{Flow Rate} = \frac{\text{Number of Pulses} \times 60}{\text{Time Interval (seconds)}}
\end{equation}
\begin{figure}[H]
    \centering
    \includegraphics[width=0.5\textwidth]{images/water-flow.jpg}
    \caption{Water flow sensor used in PiIrrigate}
    \label{fig:water-flow-sensor}
\end{figure}

In my implementation the water source is a water tank that will be connected to the irrigation system.
The tank will be equipped with a water temperature sensor. The purpose
of monitoring the water temperature is to ensure that the water is not too cold or too hot for the plants.
The water temperature sensor is a DS18B20 sensor, which is a digital temperature sensor that can be used to measure the temperature of liquids.
The DS18B20 sensor uses the 1-wire digital interface to communicate with the ESP32.
The sensor can measure temperatures from -55°C to +125°C with an accuracy of ±0.5°C.

\begin{figure}[H]
    \centering
    \includegraphics[width=0.5\textwidth]{images/water-temp.png}
    \caption{DS18B20 water temperature sensor used in PiIrrigate}
    \label{fig:ds18b20}
\end{figure}

To be able to locate each node in the field, a GPS module is used. More precisely, I used a NEO-6M GPS module
which is integrated in the LILYGO Meshtastic AXP2101 T-Beam V1.2 ESP32 LoRa development board.

\subsection {Data Transmission}
Once the data is collected from the sensors, it needs to be transmited to the gateway.
The ESP32 nodes use LoRa communication to send the data to the gateway. 
The sensor reading has a period of 10 seconds, which is the default value and it can be changed from the Web Application.
The data collected from the sensors is serialized into a sensor reading structure, 
which is then is included in a LoRa packet structure that will be 
binary serialized and sent over LoRa radio communication.
\begin{lstlisting}[language=C, caption={LoRa packet structure}]
struct LoRaPacket {
    int packetCount;      // Packet count for tracking
    SensorData sensorData; // Sensor data to be sent
    int stationMac[6]; // MAC address of the device
};
\end{lstlisting}

\begin{lstlisting}[language=C, caption={Sensor reading structure}]
struct SensorData {
    time_t timestamp;
    float temperature;
    float humidity;
    float soilMoisture;
    float rainLevel;
    float waterTemp;
    float totalWaterFlow;
    float longitude;
    float latitude;
};
\end{lstlisting}

All the values, except the totalWaterFlow, and the GPS coordinates, represent the raw
voltage values read from the sensors. I chose to sent the raw values, 
because they can be used to calculate any related values in the web application.
Allowing a better flexibility in the future.

\section{Raspberry Pi and ESP32 Gateway}
The gateway ESP23 acts like a proxy between the nodes and the Raspberry Pi.
It receives the LoRa packets from the nodes, and then it sends the data to 
the Raspberry Pi using Serial communnication by UART.

UART is an integrated cicuit that plays the most important role in serial communication.
It containts a parallel-to serial converter and a serial-to-parallel converter\cite{uderstandingUart}
\cite{laddha2013review}. The 
parrallel-to-serial converter ia used for data sent from Raspberry Pi to the ESP32,
and the serial-to-parallel converter is used for data sent from the ESP32 to the Raspberry Pi.
The UART frame format is as follows:
\begin{itemize}
    \item Start bit: 1 bit, always 0
    \item Data bits: 5 to 9 bits, usually 8 bits
    \item Parity bit: 1 bit, optional, used for error detection
    \item Stop bit: 1 or 2 bits, always 1
    \item Idle bit: 1 bit, always 1
\end{itemize}

\section {Web Api}
\subsection{Web API}
The web API is developed in \.NET ans it is responsible
for user management, data storage, and communication with the IoT Hub.
It uses Entity Framework Core to interact with the 
PostgreSQL database and SignalR to provide real-time communication.
It uses the controller pattern to handle the HTTP requests. 
I created a controller for each resource in the system,
such as users, zones, data, devices, schedules and cloud to device messages.

\subsubsection{Zone Controller}
Since the first step in the activation process is to verify the code enterede by the
user, I will start describing the zone controller.
Using this controller, the user can create a zone, activate a zone, get the IoT Device connection string, retrieve
all zonees, get a zone by id, and delete a zone. For the database interaction, I created
a zone repository that implements the IZoneRepository interface.
\begin{lstlisting}[caption={Zone Repository interface}]
public interface IZoneRepository
{
    public Task<bool> CreateZone(Zone zone);
    public Task<bool> UpdateZone(Guid Id, string Name, Guid userId);
    public Task<bool> DeleteZone(Guid Id);
    public Task<Zone> GetZoneById(Guid Id);
    public Task<Zone> GetZoneByName(string name);
    public Task<IEnumerable<Zone>> GetAllByUserId(Guid Id);
}
\end{lstlisting}

The controller is also using the IoTDeviceManager service to manage the
IoT devices in Azure IoT Hub, which implement the IioTDeviceManager interface and 
is repsonsible for creating, deleting, and checking the existence of IoT devices.

\begin{lstlisting}[caption={IoT Device manager interface}]
public interface IiotDeviceManager
{
    public Task<bool> CreateIotDevice(string zoneId);
    public Task<string> GetDeviceConnectionString(string zoneId);
    public Task<bool> DeviceExists(string zoneId);
    public Task<bool> DeleteIotDevice(string zoneId);
}
\end{lstlisting}

\subsubsection{Data Controller}
The data controller is used to retrieve data from the PostgreSQL database.
It provides endpoints for:
\begin{itemize}
    \item Get all data for a zone, including the data for all devices in that zone.
    \item Get data for a certain period of time.
    \item Get all data foor a device.
    \item Get data for a device for a certain period of time.
\end{itemize}

The data controller will use the IDataService for all the logic related to data retrieval.
For the DataService, all the database interaction is done within that service.

\subsubsection{Device Controller}
The device controller is used to manage the devices in the system.
It provides endpoints for:
\begin{itemize}
    \item Get all devices for a zone.
    \item Get all devices for a user.
    \item Get a device by id.
    \item Get a device by MAC address.
    \item Register a new device.
\end{itemize}

\subsubsection{C2D Controller}
The C2D (Cloud to Device) controller is used to manage the cloud to device messages in the system.
It provides only one endpoint to send a message to a device. The endpoint accepts a C2DMesageRequest object, 
which contains the zoneId, and an object of type C2DMethodCall, which contains the the deviceId, the method name, and a list of parameters.
\begin{figure}[H]
    \centering
    \includegraphics[width=0.8\textwidth]{images/c2d-message-request.png}
    \caption{C2D Message Request object}
    \label{fig:c2d-request}
\end{figure}

\subsubsection{Schedule Controller}
The schedule controller is used to manage the irrigation schedules in the system.
It provides endpoints for:
\begin{itemize}
    \item Get all schedules.
    \item Get a schedule by id.
    \item Create a new schedule.
    \item Update an existing schedule.
    \item Delete a schedule.
\end{itemize}

The Schedule object is defined as follows:
\begin{figure}[H]
    \centering
    \includegraphics[width=0.8\textwidth]{images/schedule.png}
    \caption{Schedule object}
    \label{fig:schedule-object}
\end{figure}

\begin{enumerate}
    \item The Id is a unique identifier for the schedule.
    \item The Request is an object of type C2DMesageRequest that contains the zoneId, deviceId, method name, and parameters.
    \item The StartTime is the time when the schedule should start.
    \item The EndTime is the time when the schedule should end.
    \item The Interval is the time interval between two consecutive executions of the schedule.
    \item The Duration is the duration of the schedule execution.
    \item The ScheduleStatus is the status of the schedule, which can be NotStarted, Running, Paused, Completed, Failed.
\end{enumerate}

The user can create a new schedule using the UI of the web application,
which will send a request to the web API to create a new schedule. 
The web API will validate the request and create a new schedule in the database.
A bragraound service will be used to check and execute the schedules. The background service will run 
every minute and check if there are any schedules that should be executed or stoped. To be more robust, the duration 
of the schedule, as well as the start and end time, will be sent to the Raspberry Pi, wich will also check if 
the schedule should be executed or stopped. This way if the Web API or the network connectivity is down, 
the Raspberry Pi will still be able to execute the schedules.

A background service is used to check the schedules. A baground services is a service that implements
the IHostedService interface and runs tasks in the background. It is used to perform long-running tasks and
it implements the StartAsync and StopAsync methods \cite{IHostedService}.


\subsubsection{User Controller}
As the user needs to be authenticated to access the web API,
I created a user controller that handles user registration and login.

\begin{lstlisting}[caption={User Service interface}]
public interface IUserService
{
    Task<UserDto> GetUserByIdAsync(Guid userId);
    Task<IEnumerable<UserDto>> GetAllUsersAsync();
    Task<bool> UpdateUserProfileAsync(Guid userId, UpdateProfileRequest request);
    Task<bool> ChangeUserRoleAsync(Guid userId, string newRole);
    Task<AuthResult> RegisterUser(RegisterRequest register);
    Task<AuthResult> LoginUser(LoginRequest request);
}
\end{lstlisting}

For accessing the user data, UserRepository is used, which implements the IUserRepository interface.
The authorization is done using JWT tokens, 
which are generated when the user logs in. To handle all the logic related to 
JWT tokens, I created a JWT service that implements the IJwtService interface.
Besides IUserRespository, and IJwtService, the UserService is also using
IPasswordHasher to hash the user passwords. This provides abstraction
for hashing and verifying passwords, 
allowing for easy integration with different hashing algorithms. For the moment,
the implementation uses SHA256 hashing algorithm, 
but it can be easily changed to any other algorithm. In the future, I plan to replace
this with Microsoft.AspNetCore.Identity, 
which provides a more secure and flexible way to handle user 
authentication and authorization and it is widely used in the \.NET community.

\begin{lstlisting}[caption={User Repository interface}]
public interface IUserRepository
{
    Task<User> GetByIdAsync(Guid id);
    Task<User> GetByEmailAsync(string email);
    Task<IEnumerable<User>> GetAllAsync();
    Task<bool> CreateAsync(User user);
    Task<bool> UpdateAsync(User user);
    Task<bool> DeleteAsync(Guid id);
}
\end{lstlisting}

\begin{lstlisting}[caption={JWT Service interface}]
public interface IJwtService
{
    string GenerateJwtToken(User user);
    bool ValidateToken(string token);
}
\end{lstlisting}

\begin{lstlisting}[caption={Password Hasher interface}]
public interface IPasswordHasher
{
    string HashPassword(User user, string password);
    bool VerifyPassword(User user,string hashedPassword, string providedPassword);
}
\end{lstlisting}

\subsubsection{Registration Flow}
The registration flow is as follows:
\begin{enumerate}
    \item The user fills the registration form in the web application.
    \item The web application sends the request to the web API. More sdpecifically, 
    it sends a POST request to the register endpoint of the user controller. The controller
    receives the request as an object of type RegisterRequest.
    \item The request is being validated by the web API.
    \item If the request is valid, the web API creates a new user in the database
    \item The web API generates a JWT token and it sends back an object of type AuthResult.
    \item The web application receives the AuthResult object and stores the JWT token in the local storage.
\end{enumerate}

The registration flow is shown in the figure below.
\begin{figure}[H]
    \centering
    \includegraphics[width=0.9\textwidth]{images/registration-flow.png}
    \caption{Registration flow}
    \label{fig:registration-flow}
\end{figure}

The AuthResult object and the RegisterRequest object are defined as follows:
\begin{figure}[H]
    \centering
    \includegraphics[width=0.8\textwidth]{images/auth-request.png}
    \caption{AuthResult and RegisterRequest objects}
    \label{fig:auth-result}
\end{figure}

\subsubsection{Login Flow}
The login flow is similar to the registration flow, but it does not create a new user.
The login flow is as follows:
\begin{enumerate}
    \item The user fills the login form in the web application.
    \item The web application sends a POST request to the login endpoint of the user controller. The controller
    receives the request as an object of type LoginRequest.
    \item The request is being validated by the web API.
    \item If the request is valid, the web API checks if the user exists in the database.
    \item If the user exists, the web API verifies the password and generates a JWT token.
    \item The web API sends back an object of type AuthResult.
    \item The web application receives the AuthResult object and stores the JWT token in the local storage.
\end{enumerate}

The login flow is shown in the figure below.
\begin{figure}[H]
    \centering
    \includegraphics[width=0.9\textwidth]{images/login-flow.png}
    \caption{Login flow}
    \label{fig:login-flow}
\end{figure}