\chapter{Conclusions}
\thispagestyle{pagestyle}

\begin{code}
	\begin{lstlisting}[language=Java]
public class Client {
	public static void main(String[] args) {
		Animal tiger = new Tiger();
		Animal parrot = new Parrot();
		tiger.breed(parrot);
	}
}
	\end{lstlisting}
	\caption{Subtype polymorphism example \cref{code:polym2}}
	\label{code:polym3}
\end{code}

The paper will end with a chapter of conclusions. It will contain the main results of the work and their practical implications. In the case of diploma projects, the main synthetic data obtained from the design process will be mentioned.

\section{Bibliography}
At the end of the paper will be given a list of references for the scientific texts consulted during the work. All sources will be listed, including those on the internet. These will be referenced in the text and listed in alphabetical order, as in the examples below.

\textbf{IMPORTANT:} Citations are not entered manually, but by using the biblatex library. References can be generated with various external tools in .bib format and then inserted into the project structure in bibliography.bib. Further reading:
\begin{enumerate}
    \item Docs: \url{https://mirrors.nxthost.com/ctan/macros/latex/contrib/biblatex/doc/biblatex.pdf}
    \item Convert DOI to .bib: \url{https://www.doi2bib.org/}
\end{enumerate}


Bibliography should include all literature titles that have served as a basis for documentation, i.e. authors who have been quoted in the text, in all chapters of the thesis paper. 

The Faculty of Automation and Computers requires the use of the IEEE citation style (details \url{https://ieee-dataport.org/sites/default/files/analysis/27/IEEE\%20Citation\%20Guidelines.pdf}), used primarily in scientific publications in the field of IT. The three important parts of the reference are:
\begin{enumerate}[leftmargin=2cm,topsep=1.15pt,itemsep=1.15pt,partopsep=1.15pt,parsep=1.15pt,label=\alph*.]
   \item The name of the author indicated as the first initial of the first name, then the full name.
   \item The title of the article, the patent, the conference paper, etc., in quotation marks.
   \item The title of the magazine or book in italics.
How the reference is written depends on the type of publication, please follow the instructions at the link above carefully.
\end{enumerate}
	 
Each citation should be noted in the text using simple sequential numbers. A number in square brackets, placed in the text of the report, indicates the specific reference. Citations are numbered in the order in which they appear. Once a source has been cited, the same number is used in all subsequent references in the text. No distinction is made between electronic and printed sources, except for the details of the cited references.


Each reference number must be enclosed in square brackets on the same line as the text, before any punctuation mark, with a space before the parentheses.

Examples:
\begin{enumerate}[leftmargin=2cm,topsep=1.15pt,itemsep=1.15pt,partopsep=1.15pt,parsep=1.15pt,label=\alph*.]
   \item ". . .the end of my research [13]."
   \item "The theory was first introduced in 1987 [1]."
\end{enumerate}

The list of references in the bibliography is composed of all the sources used to document the paper and is made in the numerical order of the citation in the text and not in alphabetical order of the authors.

The identical insertion of a sentence or paragraph shall be made by including the page from the source used, but also by quotation marks and the use of Italics; for sources taken from the Internet, the page addresses shall be included; in the final bibliographic list the works shall be entered in the alphabetical order of the authors' names. For collective works, the rule of alphabetical order applies to the first author. 

If websites, magazines or articles are quoted, three asterisks will appear before, and then information on the volume, the issue, the pages consulted, the exact website address of the article, the date of the site visit and the date of downloading, as well as the date of the accessing. Web page addresses will be entered at the end of the list.

The bibliographical sources the author of which cannot be mentioned should be specified as "***" followed by the name of the article and/or book, the publishing house and the place of appearance (for books), the volume, the issue, the first and last page of the quoted work, and the year of appearance. 

*** https://ro.wikipedia.org/wiki/Motor accessed February 2022

Example: \label{example:citation} Einstein \cite{einstein}, mentioning \enquote{The intuitive mind is a sacred gift.}

\section{Authenticity Declaration}
The last page of the thesis paper shall contain the „Statement regarding the authenticity of the thesis paper”, in handwriting, filled in according to the UPT’s requirements. The Statement shall be downloaded from the web, at:

\url{http://www.upt.ro/img/files/Regulamente_UPT/2020/Declaratie_de_autenticitate_UPT_2020.pdf}